\documentclass[]{style}

\newcommand{\printbibsectionkeyword}[3]{
  \begin{refsection}
    \newrefcontext[sorting=chronological]%
    \nocite{*}
    \printbibliography[type={#1}, title={#2}, keyword={#3}, heading=bibheading]
  \end{refsection}
}

\usepackage{fontawesome}
\usepackage{array}
\usepackage{hyperref}
\usepackage{tabularx}
\usepackage[ddmmyyyy]{datetime}


\newcolumntype{Y}{>{\centering\arraybackslash}X}


\addbibresource{bibliography.bib}
\addbibresource{misc.bib}


\begin{document}

\header{Federico A. }{Galatolo}{Information Engineering PhD}


\begin{aside}
\section{{\color{white}.}}
\includegraphics[width=4cm]{me}
\section{Contatti}
\begin{tabular}{m{1cm}m{3cm}}
    \faicon{globe}& \href{https://galatolo.me}{galatolo.me} \\
    \faicon{mobile} & \href{tel:+393939592790}{+393939592790} \\
    \faicon{at} & \href{mailto:federico.galatolo @ing.unipi.it}{federico.galatolo @ing.unipi.it} \\
    \faicon{paper-plane} & \href{https://telegram.me/galatolo}{galatolo} \\
    \faicon{github} & \href{https://github.com/galatolofederico}{galatolofederico} \\
\end{tabular}
\section{Dove}
\begin{tabular}{m{1cm}m{3cm}}
    \faicon{map} & Pisa (56122) \\
    \faicon{map-pin} & Largo Lucio Lazzarino 1 \\
    \faicon{building} & Building A, Floor 2, Room 113
\end{tabular}
\section{Lingue}
\begin{tabular}{m{1cm}m{3cm}}
  \faicon{comments-o} & Italian \\
  \faicon{comments-o} & English 
\end{tabular}
\section{Programmazione}
\begin{tabular}{m{1cm}m{3cm}}
  \faicon{code} & Python \\
  \faicon{code} & Javascript \\
  \faicon{code} & C++ \\
  \faicon{code} & Java 
\end{tabular}
\end{aside}

\section{formazione}

\begin{entrylist}

\entry
{2007--2013}
{Diploma {\normalfont Liceo Scientifico}}
{Liceo Scientifico G. Marconi, Grosseto (Italy)}

\entry
{2013--2016}
{Laurea Triennale {\normalfont in Ingegneria Informatica}}
{University of Pisa, Pisa (Italy)}

\entry
{2016--2018}
{Laurea Magistrale {\normalfont in Computer Engineering}}
{University of Pisa, Pisa (Italy)}

\entry
{2018--today}
{PhD {\normalfont in Ingegneria dell'Informazione}}
{University of Pisa, Pisa (Italy)}

\end{entrylist}

\section{lavoro}
\begin{entrylist}
\entry
{2021--2023}
{Assegno di Ricerca {\normalfont nel progeto SERICA}}
{University of Pisa, Pisa (Italy)}

\entry
{2023--Now}
{Ricercatore a Tempo Determinato Junior}
{University of Pisa, Pisa (Italy)}

\end{entrylist}

\section{progetti accademici}
\begin{entrylist}

\entry
{PRA 2022\_101}
{Sistemi di Supporto alle Decisioni per le Reti Territoriali nella Gestione dei Servizi Ecosistemici}
{Università di Pisa}
{Tecnologie di intelligenza artificiale per il monitoraggio delle risorse idriche utilizzando dati ambientali e satellitari.}

\entry
{Secure B2C}
{POR FESR 2014-2020}
{Italia}
{Sviluppo di prototipi di intelligenza artificiale per la sicurezza delle reti aziendali e l'architettura hardware/software per dispositivi intelligenti di pagamento elettronico. Pubblicato e presentato a una conferenza internazionale.}

\entry
{MUR-FISR 2019}
{SERICA}
{Università di Pisa}
{Progettazione di una biblioteca digitale utilizzando l'intelligenza artificiale per l'analisi del testo e mappe interattive. Contributo significativo ai componenti di intelligenza artificiale e supporto nello sviluppo di questi sistemi. Pubblicato e presentato a una conferenza internazionale.}

\entry
{H2020}
{EXPERIENCE}
{Centro E. Piaggio}
{Integrazione della realtà virtuale e componenti psicologiche per simulazioni virtuali. Sviluppo di una rete neurale per la regressione del segnale fNIRS.}

\entry
{Ki-Foot}
{POR FESR 2014-2020}
{Italia}
{Prototipo per il monitoraggio dell'andatura basato sull'integrazione di sensori nelle calzature. Collaborazione su tesi di laurea.}

\entry
{PRA 2018\_81}
{Progetto di Ricerca Universitaria}
{Università di Pisa}
{Soluzioni basate sull'intelligenza artificiale per problemi di ricerca specifici. Pubblicato e presentato a una conferenza internazionale.}

\entry
{SCIADRO}
{Sicurezza Ambientale e Protezione con Droni}
{Italia}
{Utilizzo di sciami di droni per rilevare perdite di gas e incendi precoci. Sviluppi basati sulla stigmergia computazionale e algoritmi bio-ispirati. Pubblicato e presentato a una conferenza internazionale.}

\end{entrylist}

\section{pubblicazioni}

\printbibsection{article}{Articoli in Journal Internazionali con peer-review} 
\printbibsection{inproceedings}{Articoli in Conferenze Internazionali con peer-review} 

\section{ruoli accademici}

\printbibsectionkeyword{misc}{Per Journal Internazionali}{journalrole}
\printbibsectionkeyword{misc}{Per Conferenze Internazionali}{conferencerole}

\section{progetti open-soruce}

\printbibsectionkeyword{misc}{Progetti personali open-source}{project}



\end{document}
