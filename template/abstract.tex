My research centers on AI-based architectural solutions, with expertise in machine and deep learning, NLP, computer vision, generative architectures, reinforcement learning, and multi-objective optimization. During my PhD, I developed bio-inspired neural architectures, integrated neural networks for multimedia, and tackled differentiation instability in reinforcement learning. I enhanced neural network classification and devised new inference pipelines for text-based math problem-solving. In the SERICA project, I handled information extraction from scans and software architecture. I also worked on neural network-driven structural monitoring and satellite data interpolation.   

I well versed in neural architectures like CNNs (VGG, ResNet, etc.), temporal networks (RNN, Transformers), generative models (GAN, StyleGAN), and semi-supervised models (AutoEncoder, VAE, etc.), using cutting-edge training methods like Contrastive Language-Image Pre-training.   

I teach Computational Intelligence and Deep Learning at the University of Pisa for the AI and Data Engineering master's program. I'm a founding member of the MLPI research group and an advocate for Free Software, contributing to open-source projects. 